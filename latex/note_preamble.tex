\usepackage{amssymb}
\usepackage{amsmath}
\usepackage{xcolor}
\usepackage[T1]{fontenc}

\newcommand*{\half}{\frac{1}{2}}

\newcommand*{\abs}[1]{\left\lvert#1\right\rvert}
\newcommand*{\norm}[1]{\left\lVert#1\right\rVert}
\newcommand*{\normsq}[1]{\norm{#1}^2}
\newcommand*{\given}{\mid} % Just a naming alias
\newcommand*{\inv}[1]{#1\raisebox{1.15ex}{$\scriptscriptstyle-\!1$}}
\newcommand*{\sR}{\mathbb{R}}
\newcommand*{\Ex}{\mathbb{E}}
\newcommand*{\Prob}{\mathbf{P}}

\newcommand*{\lline}{\rule{\linewidth}{0.2pt}}

% https://tex.stackexchange.com/questions/5223/command-for-argmin-or-argmax
\DeclareMathOperator*{\argmax}{arg\,max}
\DeclareMathOperator*{\argmin}{arg\,min}

% https://tex.stackexchange.com/questions/268766/curly-braces-in-math-mode
% Allows the use of \set{..} in math mode
\usepackage{mathtools}
\DeclarePairedDelimiter\set\{\}

% use paragraph spacing and remove paragrpah indents
\usepackage{parskip}
\setlength\parindent{0pt}
