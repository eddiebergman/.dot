\usepackage{amsmath}
\usepackage{amssymb}
\usepackage[T1]{fontenc}
\usepackage{xcolor}


\usepackage{url}

\usepackage{amsthm}
\usepackage{mdframed}
\usepackage{thmtools}
\usepackage{hyperref}
\usepackage{marginnote}

\hypersetup{
    colorlinks,
    linkcolor={red!50!black},
    citecolor={blue!50!black},
    urlcolor={blue!80!black}
}

\newcommand*{\half}{\frac{1}{2}}

\newcommand*{\abs}[1]{\left\lvert#1\right\rvert}
\newcommand*{\norm}[1]{\left\lVert#1\right\rVert}
\newcommand*{\normsq}[1]{\norm{#1}^2}
\newcommand*{\given}{\mid} % Just a naming alias
\newcommand*{\inv}[1]{#1\raisebox{1.15ex}{$\scriptscriptstyle-\!1$}}
\newcommand*{\sR}{\mathbb{R}}
\newcommand*{\Ex}{\mathbb{E}}
\newcommand*{\Prob}{\mathbf{P}}
\newcommand*{\idx}[1]{{\left[#1\right]}}
\newcommand*{\union}{\cup}
\newcommand*{\intersection}{\cap}

\newcommand*{\lline}{\rule{\linewidth}{0.2pt}}

% https://tex.stackexchange.com/questions/5223/command-for-argmin-or-argmax
\DeclareMathOperator*{\argmax}{arg\,max}
\DeclareMathOperator*{\argmin}{arg\,min}

% https://tex.stackexchange.com/questions/268766/curly-braces-in-math-mode
% Allows the use of \set{..} in math mode
\usepackage{mathtools}
\DeclarePairedDelimiter\set\{\}

% use paragraph spacing and remove paragrpah indents
\usepackage{parskip}
\setlength\parindent{0pt}

% General reference
% https://tex.stackexchange.com/questions/110244/using-mdframed-style-within-declaretheoremstyle-to-change-title
\declaretheoremstyle[
    headfont=\normalfont\bfseries,
    notefont=\normalfont\itshape,
    bodyfont=\normalfont,
    notebraces={(}{)},
    headpunct=,
    mdframed={linecolor=black!50}
]{blackboxstyle}
\declaretheorem[numbered=no, name=, style=blackboxstyle]{blackbox}
\declaretheorem[numbered=no, name=, style=blackboxstyle]{bigref}

\declaretheoremstyle[
    headfont=\normalfont\bfseries,
    notefont=\normalfont,
    bodyfont=\normalfont,
    notebraces={(}{)},
    headpunct=.,
]{definitionstyle}
\declaretheorem[name=Defintion, style=definitionstyle, numberwithin=section]{definition}

% For use with inkscape-figures
% https://github.com/gillescastel/inkscape-figures
\usepackage{import}
\usepackage{pdfpages}
\usepackage{transparent}

\newcommand{\incfig}[2][1]{%
    \def\svgwidth{#1\columnwidth}
    \import{./figures/}{#2.pdf_tex}
}

\pdfsuppresswarningpagegroup=1
